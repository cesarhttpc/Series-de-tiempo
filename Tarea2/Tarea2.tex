\documentclass[a4paper, 11pt]{article}
\usepackage{comment}
\usepackage{lipsum} 
\usepackage{fullpage} %cambiar margen
\usepackage[a4paper, total={7in, 10in}]{geometry}

\usepackage{amssymb,amsthm} 
\usepackage{amsmath}
\newtheorem{theorem}{Theorem}
\newtheorem{corollary}{Corollary}
\usepackage{graphicx}
\usepackage{tikz}
\usetikzlibrary{arrows}
\usepackage{verbatim}
%\usepackage[numbered]{mcode}
\usepackage{float}
\usepackage{tikz}
\usetikzlibrary{shapes,arrows}
\usetikzlibrary{arrows,calc,positioning}
\usepackage{mathpazo} %tipo de letra 
\usepackage[utf8]{inputenc} %codificación
\usepackage[T1]{fontenc} %digitación de tildes y ñ
\usepackage[spanish]{babel} %paquete de soporte español

\tikzset{
	block/.style = {draw, rectangle,
		minimum height=1cm,
		minimum width=1.5cm},
	input/.style = {coordinate,node distance=1cm},
	output/.style = {coordinate,node distance=4cm},
	arrow/.style={draw, -latex,node distance=2cm},
	pinstyle/.style = {pin edge={latex-, black,node distance=2cm}},
	sum/.style = {draw, circle, node distance=1cm},
}
\usepackage{xcolor}
\usepackage{mdframed}
\usepackage[shortlabels]{enumitem}
\usepackage{indentfirst}
\usepackage{hyperref}

\usepackage{listings}
\lstset{literate=
  {á}{{\'a}}1
  {é}{{\'e}}1
  {í}{{\'i}}1
  {ó}{{\'o}}1
  {ú}{{\'u}}1
  {Á}{{\'A}}1
  {É}{{\'E}}1
  {Í}{{\'I}}1
  {Ó}{{\'O}}1
  {Ú}{{\'U}}1
  {ñ}{{\~n}}1
  {ü}{{\"u}}1
  {Ü}{{\"U}}1
}

\lstdefinestyle{customc}{
  belowcaptionskip=1\baselineskip,
  breaklines=true,
  frame=L,
  xleftmargin=\parindent,
  language=Python,
  showstringspaces=false,
  basicstyle=\footnotesize\ttfamily,
  keywordstyle=\bfseries\color{green!40!black},
  commentstyle=\itshape\color{purple!40!black},
  identifierstyle=\color{blue},
  stringstyle=\color{orange},
}

\lstdefinestyle{customasm}{
  belowcaptionskip=1\baselineskip,
  frame=L,
  xleftmargin=\parindent,
  language=[x86masm]Assembler,
  basicstyle=\footnotesize\ttfamily,
  commentstyle=\itshape\color{purple!40!black},
}

\lstset{escapechar=@,style=customc}



\renewcommand{\thesubsection}{\thesection.\alph{subsection}}

\newenvironment{problem}[2][Ejercicio]
{ \begin{mdframed}[backgroundcolor= red!50] \textbf{#1 #2} \\}
	{  \end{mdframed}}

% Define solution environment
\newenvironment{solution}
{\textcolor{blue}{\textbf{\textit{Solución:\\\noindent}}}}


\renewcommand{\qed}{\quad\qedsymbol}

% \\	
\begin{document}
	\noindent
	%%%%%%%%%%%%%%%%%%%%%%%%%%%%%%%%%%%%
	
	\begin{minipage}[b][1.2cm][t]{0.8\textwidth}
		\large\textbf{César Isaí García Cornejo} \hfill \textbf{Tarea 1}  \\
		cesar.cornejo@cimat.mx \hfill \\
		\normalsize Series de Tiempo \hfill Semestre 3\\
	\end{minipage}
	
	\hspace{14.4cm}
	\begin{minipage}[b][0.03cm][t]{0.12\linewidth}
		
		\vspace{-2.2cm}
		%%%La Ruta depeendera de donde este alojado el main y la imagen
		\includegraphics[scale=0.3]{Figures/EscudoCimat.png}
	\end{minipage}
	
	\noindent\rule{7in}{2.8pt}
	
	%%%%%%%%%%%%%%%%%%%%%
	%%%%%%%%%%%%%%%%%%%%%%%%%%%%%%%%%%%%%%%%%%%%%%%%%%%%%%%%%%%%%%%%%%%%%%%%%%%%%%%%%%%%%%%%%%%%%%%%%%%%%%%%%%%%%%%%%%%
	% Problem 1
	%%%%%%%%%%%%%%%%%%%%%%%%%%%%%%%%%%%%%%%%%%%%%%%%%%%%%%%%%%%%%%%%%%%%%%%%%%%%%%%%%%%%%%%%%%%%%%%%%%%%%%%%%%%%%%%%%%%%%%%%%%%%%%%%%%%%%%%%
	\setlength{\parskip}{\medskipamount}
	\setlength{\parindent}{0pt}
 %///////////////////////////////////////////////////
\begin{problem}{1}
    Considera un proceso estacionario $AR(1)$ dado por 
    \begin{align*}
        Y_t = \frac{1}{2}Y_{t-1} + e_t 
    \end{align*}
    donde $e_t$ son no-correlacionados $(0,\sigma^2)$. Define 
    \begin{align*}
        v_t = Y_t - 2Y_{t-1}.
    \end{align*}
    \begin{enumerate}
        \item Demuestra que el residual $v_t$ es una sucesión de v.a. no -correlacionadas $(0, \sigma_v^2)$. ?` Cuál es la varianza de $v_t$ ? ?` Quién tiene más varainza $e_t$ o $v_t$ ?
        \item Demuestra que $e_t$ no está correlacionado con $Y_{t-1}$ y que $v_t$ está correlacionado con $Y_{t-1}$ .
        \item Expresa $Y_t$ como una media móvil $MA(\infty)$.
    \end{enumerate}

    La raíz de la ecuación característica de la ecuación en diferencias sociada a $Y_t = 2 Y_{t-1} + v_{t }$ es 2 (i.e. es mayor que uno). Entonces, para $ Y_t = \alpha_1 Y_{t-1} + v_t$, las condiciones $v_t$ no-correlacionadas $(0, \sigma^2_v )$ y $|a_1| > 1$ no implican que $Y_t$ es no-estacionario.

    En este ejemplo preferimos la representacion $Y_t = \frac{1}{2}Y_{t-1} + v_t$ pues, como demostrarás en $a)-c)$, el error tiene menor varianza y no está correlacionado con 
    
    
    
    
    
\end{problem}

\begin{problem}{2}
    Sea $Y_t$ una serie de tiempo definida como 
    \begin{align*}
        Y_t = \beta_0 + \beta_1 t + X_t \:\:\:\:\:\: t = 1,2, \cdots
    \end{align*}
    donde
    \begin{align*}
        X_t = e_t + 0.6 e_{t-1},
    \end{align*}
    con $\beta_0,\beta_1$ fijos y $\{e_t: t\in \mathbb{N}\cup \{0\} \} $ distribuidas $N(0,sigma^2)$ Construye la media y la función de covarianza para $Y_t$.
\end{problem}

\begin{problem}
    Sean $X_i \sim N(0, \sigma^2) \: \: i = 1,2,\cdots$ independientes y sea $\bar{X}_n = n^{-1} \sum _{i=1}^n X_i .$
    \begin{enumerate}
        \item Se sabe que $\mu \neq 0$. ?` Cómo aproximarías la distribución de $\bar{X}_n^2 $ en muestras grandes? 
        \item Se sabe que $\mu = 0$. ?` Cómo aproximarías la distibución de $\bar{X}_n^2$ en muestras grandes ?
        \item Comenta qué pasa si quitamos el supuesto de independencia en los incisos anteriores. 
    \end{enumerate} 
    Explica con detalle los procedimientos y asegúrate de que no se den distribuciones límites degenerads. Este ejercicio es para recordar los procedimientos más básicos para variables aleatorias.
\end{problem}

\begin{problem}{4}
    Monstrar que si $m^p + a_1 m^{p-1} + \cdots + a_p = 0 $ tiene todas sus raíces menores que uno en módulo, entonces $1+ a_1 q + \cdots + a_p q^p = 0$ tiene todas sus raíces mayoures que uno en módulo. \textit{Hint: si r es una raíz del primer polinomio, es ?` es 1/r una raíz del segundo?}    
\end{problem}

% \begin{problem}{5}
%     Si $m_1, \cdots, m_p$ son las raíces de $ m^p - \sum _{i=1}^p \alpha_i m^{p-i} =  0$, entonces
%     \begin{enumerate}
%         \item $\sum _{i=1}^p \alpha_i = 1$ si y sólo si al menos una raíz es igual a 1.
%         \item Si todas las raíces en módulo son menores que 1, entonces $\sum_{i=1}^p \alpha_i <1.
%     \end{enumerate}
% \end{problem}





\end{document}